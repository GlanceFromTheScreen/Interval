\documentclass[a4paper,12pt]{article}

\usepackage[T2A]{fontenc}
\usepackage[utf8]{inputenc}
\usepackage[english, russian]{babel}

\usepackage{amsthm, amsmath, amssymb}

\usepackage{graphicx}

\usepackage{hyperref}
\usepackage{float}

\usepackage[a4paper, total={6.5in, 10in}]{geometry}


\begin{document}
% Title page 
\begin{titlepage}
    \begin{center}
        \textsc{
            Санкт-Петербургский политехнический университет имени Петра Великого \\[5mm]
            Физико-механический институт\\[2mm]
            Высшая школа прикладной математики и физики            
        }   
        \vfill
        \textbf{\large
            Отчет по лаборабороной работу №1\\
            по дисциплине "Интервальный анализ"\\[3mm]
        }                
    \end{center}

    \vfill
    \hfill
    \begin{minipage}{0.5\textwidth}
        Выполнил: \\[2mm]   
		Студент: Хламкин Евгений \\
		Группа: 5030102/00201\\
    \end{minipage}

	\hfill
	\begin{minipage}{0.5\textwidth}
		Принял: \\[2mm]
		к. ф.-м. н., доцент \\   
		Баженов Александр Николаевич
	\end{minipage}

    \vfill
    
    \begin{center}
    Санкт-Петербург\\
    2023 г.\\
    \end{center}
\end{titlepage}

\tableofcontents
\newpage

\section{Постановка задачи}
Найти минимальную $\delta$, чтобы матрица была особенной\\
Пусть $\textbf{X}$ - интервальная матрица и
\begin{equation}
\mathrm{mid}(\textbf{X})=
\begin{pmatrix}
1.05 & 1 \\
0.95 & 1
\end{pmatrix} 
\end{equation}
Необходимо рассмотреть матрицы $X_1$ и $X_2$ для задачи регрессии и томографии соответственно:
\begin{equation}
\mathbf{X_1}=
\begin{pmatrix}
[1.05-\delta, 1.05+\delta] & [1, 1] \\
[0.95-\delta, 0.95+\delta] & [1, 1]
\end{pmatrix} \;\;\;
\end{equation}
\begin{equation}
\mathbf{X_2}=
\begin{pmatrix}
[1.05-\delta, 1.05+\delta] & [1-\delta, 1+\delta] \\
[0.95-\delta, 0.95+\delta] & [1-\delta, 1+\delta]
\end{pmatrix}
\end{equation}\\



\section{Теория}
\subsection{Определения}
\begin{itemize}
    \item Середина матрицы $\mathrm{mid}(\textbf{A})=\{A\;|\;a_{ij}=\mathrm{mid}(\textbf{a}_{ij})\}$
    \item Радиус матрицы $\mathrm{rad}(\textbf{A})=\{A\;|\;a_{ij}=\mathrm{rad}(\textbf{a}_{ij})\}$
    \item Матрица $\textbf{A}\in \mathbb{IR}$ называется особенной, если $\exists A \in \textbf{A} : det(A) = 0$.
    \item Числа $\sigma_1...\sigma_k$, равные квадратным корням из собственных значений матрицы $AA^T$, называется cингулярными числами матрицы $A$.
    \item Множество вершин интревальной матрицы\\ $\mathrm{vert}(\textbf{A})=\{A\in\mathbb{IR}^{m\times n} \;|\; A=(a_{ij}) \, a_{ij}\in\{\underline{\textbf{a}}_{ij}, \overline{\textbf{a}}_{ij}\}\}$
\end{itemize}

\section{Реализация}

\begin{enumerate}
    \item Если интервал симметричен при произвольном $\delta$, то ответ: $0$
    \item Иначе - применияем метод дихотомии. Устанавливаем на нулевой итерации значение $\delta = 0$, затем с шагом $\epsilon$ движемся вправо. Если при этом $0 \in DET$, то возвращаемся и уменьшаем шаг.
\end{enumerate}


\section{Результаты}
\begin{enumerate}
    \item Случай 1 \\
        Видим, что  $0 \in DET$, а также $mid(X_1) \neq 0$, значит, переходим к пункту 2 описанного алгоритма.
        В результате получаем $min (\delta) = 0.025$. В таком случае $DET(X_1) = [2.220 *10^{-16}, 0.2]$. Левый конец с точностью до машинного эпсилон равен нулю

     \item Случай 2 \\
        Видим, что $0 \in det X_2$, а также $midX_2  \neq 0$,
        значит, переходим к пункту 2 алгоритма.
        В результате получаем min $\delta = 0.05$. В таком случае $DET(X_2) = [1.110 * 10^{−16}
        , 0.2]$. Левый конец $DET{X_2}$ с
        точностью до машинного эпсилон равен нулю



\section{Вывод}
Данные матрицы $X_1$,$\;X_2$  являются  неособенной при $\delta < 0.051285$ и $\delta \le 0.025$.

\section{Ссылка на Github}

\href{https://github.com/GlanceFromTheScreen/Interval}{GitHub}.

\end{document}


